\documentclass{article}
\usepackage[UTF8]{ctex}
\usepackage{booktabs}
\usepackage{multirow}
\usepackage{graphicx}
\usepackage{tikz}
\usetikzlibrary{shapes,arrows,chains}
\usepackage{pgfplots}
\usepackage{pgf-pie}
\pgfplotsset{compat=1.18}
\usepackage{hyperref}
\usepackage{tocloft} % 目录格式

% 目录格式调整
\renewcommand{\cftsecleader}{\cftdotfill{\cftdotsep}}
\renewcommand{\cftsubsecleader}{\cftdotfill{\cftdotsep}}

\begin{document}

\title{胖猫自杀事件多平台舆情分析报告}
\author{\vspace{0.5cm} \normalsize 第二组 \\ \vspace{0.2cm} \normalsize \today}
\date{}
\maketitle

\tableofcontents
\newpage

\section{引言}
2024年4月至5月期间,胖猫自杀事件引发了社会各界的广泛关注和激烈讨论,各大社交媒体平台上纷纷涌现出大量与该事件相关的信息。作为一个涉及个人悲剧、网络舆论对立和品牌危机的复杂事件,本研究旨在对胖猫自杀事件期间各主要平台舆情的变化进行全面而细致的分析。借助于Python爬虫、Sina Opinion数据采集工具以及各平台公开数据,我们对事件发生前后不同时间节点的信息传播、互动量和情感倾向进行了系统统计,并利用数据可视化手段展示各阶段舆情走势。这一研究不仅可以揭示信息扩散过程中的关键节点和转折点,还能为今后相似公共事件的舆情管理和危机应对提供有益的经验和理论支持。

\section{摘要}
“胖猫事件”由一起私人情感纠纷演变为全国性网络风暴,暴露了网络空间治理的核心问题。2024年4月,湖南青年刘某(网名“胖猫”)因与女友谭某分手在重庆跳江轻生。其姐姐通过曝光隐私、煽动舆论,将事件包装为“捞女诈骗”事件,引发网络暴力、外卖祭奠乱象、谣言传播及性别对立等连锁反应。事件造成94.6吨外卖垃圾、公众信任危机及平台治理漏洞等后果,凸显了网络民粹主义与流量经济的失控。

本文围绕事件期间各平台舆情变化展开分析,收集微博、抖音等多平台数据,其中微博信息量达275万余条,占总量的43.85\%,互动量峰值突破2.85亿次。情感分析显示,网民情绪以愤怒(约24\%)和悲伤为主。研究结合定量数据与定性文本分析,梳理时间线、探讨关键意见领袖作用及媒体报道影响,总结传播规律,为舆情引导和危机公关提供参考。

\section{详细时间线与事件节点}
胖猫自杀事件的时间线较为明确,主要涉及以下几个关键节点:
\begin{itemize}
    \item \textbf{2024年4月11日}:胖猫在重庆一座桥上结束了自己的生命。根据相关资料,事发前他曾连续工作数小时,情绪极度低落,并在事发当日将最后一笔6.6万元转账给其女友,寄望挽回关系。
    \item \textbf{2024年5月2日}:家属公开发布胖猫生前与女友的聊天记录,引发网友对事件真实性和情感操控的进一步质疑。此举使得事件信息量骤然激增,成为全网热议的话题。
    \item \textbf{2024年5月4日}:随着网红“Yin Shihang 77”在微博上发布支持胖猫家属的帖子,当日全平台相关信息数量达到了历史高峰,微博平台信息量一度突破145万条,互动量达到2.85亿次左右。
    \item \textbf{2024年5月19日}:警方公布最终调查结果,并对事件中涉及的部分谣言做出澄清。此举在一定程度上引导了网络舆论理性回落,平息了部分极端情绪。
\end{itemize}
除此之外,事件还引发了多起品牌危机公关。例如,部分餐饮品牌因“悼念胖猫”行动处理不妥而面临公关风暴,麦当劳越南分部因不当广告内容道歉,一定程度上说明了公众在面对悲剧时对商业宣传敏感度的提高。

\subsection{时间线流程图}
\begin{center}
\begin{tikzpicture}[node distance=2cm]
\node (A) [rectangle, draw] {2024年4月11日:胖猫自杀};
\node (B) [rectangle, draw, below of=A] {2024年5月2日:家属公开聊天记录};
\node (C) [rectangle, draw, below of=B] {2024年5月4日:舆情高峰};
\node (D) [rectangle, draw, below of=C] {2024年5月19日:警方公布结果};
\node (E) [rectangle, draw, below of=D] {后续品牌危机公关};
\draw[->] (A) -- (B);
\draw[->] (B) -- (C);
\draw[->] (C) -- (D);
\draw[->] (D) -- (E);
\end{tikzpicture}
\end{center}

\section{舆情概述}
胖猫事件在短时间内引发了叠加性舆情效应,各类讨论从最初的个案悲剧迅速扩散到对网络暴力、情感操控以及性别关系等层面的广泛议题。主要特点包括:
\begin{itemize}
    \item \textbf{信息量巨大且扩散迅速}:通过Sina Opinion工具采集的数据表明,仅微博平台上的相关信息数量就达到了275万余条,占全部相关信息的43.85\%,而互动量在高峰期突破2.85亿次,说明该事件极大地牵动了网民情绪。
    \item \textbf{情感倾向极端化}:情感分析显示,除中性情绪外,愤怒和悲伤占据了绝大部分讨论情绪,其中大约24\%的官方及大众报道呈现出明显负面情绪。网民对女方的指责和对社会风气的担忧均在舆论中有所体现。
    \item \textbf{意见领袖与网红的引导作用}:网红“Yin Shihang 77”的介入是事件舆论飙升的重要转折点。其在社交平台上的支持发声不仅拉动了事件讨论量,还将事件从局部小圈子推向全网,进一步激化了舆情。
    \item \textbf{跨平台信息传播及多维讨论}:除了微博,抖音、微信、论坛等平台也有大量网民参与讨论。虽然不同平台的具体数据存在差异,但总体来看,事件在多平台上的传播具有“扩散—聚合—再扩散”的特点,并衍生出大量二次传播和深度讨论。
    \item \textbf{商业与舆论交叉影响}:事件的扩散不仅冲击到公众关注,还引发了品牌公关危机。例如,部分餐饮品牌因不当的“悼念”行为而陷入舆论漩涡,部分企业如麦当劳越南分部被迫公开道歉,体现了公众对于悲剧商业化营销的高度敏感性。
\end{itemize}
总体而言,胖猫事件不仅反映了单一悲剧的社会效应,更揭示了当前网络舆情传播中存在的盲目性和群体极化问题,为相关部门的治理和企业的危机应对敲响了警钟。

\section{数据收集与分析方法}
本研究主要采用以下几种方法对胖猫事件期间的舆情进行分析:
\begin{itemize}
    \item \textbf{数据采集}:利用Python爬虫技术和第三方工具(如Sina Opinion)对微博、抖音及其他平台上的相关话题进行了数据抓取。筛选条件包括:发布时段(2024年4月至5月)、互动量不少于300个“点赞”、以及信息内容与事件关联度高等标准。最终从微博平台筛选出约511篇高互动原始信息,并将相关数据构建成结构化数据集。
    \item \textbf{情感分析}:在数据预处理阶段,利用Python进行文本情感词汇统计,对每条信息的情感倾向进行评分,其中分数范围设置为:负面情绪为-10至0,零分表示中性,正面情绪为0至20。通过统计正面、负面与中性情感词的数量,进一步划分出情绪占比,并形成详细的情感分布图表。
    \item \textbf{时序分析}:将事件节点与数据采集时段进行对比,利用折线图展示不同时间节点上的信息量和互动量波动情况,探索关键意见领袖发布信息对整体舆情走势的影响。例如,重点关注5月4日网红发声前后的数据变化。
    \item \textbf{内容主题分析}:对数据中涉及的讨论主题进行归类,梳理出关于情感操控、网络暴力、性别争议以及商业危机公关等主要议题,并通过关键词云及主题词频统计图进行可视化展示。通过主题词频分析,进而剖析信息发布者主要关注的问题点与批评意见。
\end{itemize}
\textbf{方法优势与局限}:本研究采用的定量与定性相结合的方法能够较为全面地捕捉和再现事件发生过程中的舆情变化,但由于数据采集中存在部分平台信息获取的局限性,仍存在数据遗漏或噪音问题。为此,后续研究中可考虑增加跨平台数据对比与深度文本语义分析方法,进一步提高数据的准确性和结论的稳健性。

\section{各平台舆情变化与数据分析}
\subsection{微博平台舆情变化分析}
微博平台作为国内最大的社交媒体之一,在此次事件中的信息传播和公众互动起到了重要作用。根据采集的数据统计,微博平台上与胖猫事件相关的信息达到2752528余条,占所有平台信息总量的43.85\%。详细分析如下:
\begin{itemize}
    \item \textbf{信息量与互动量的变化}:在事件初期,讨论主要集中在局部社群及游戏圈内,随着相关关键时间节点(如5月2日家属聊天记录公开、5月4日意见领袖发声),微博信息量和互动量急剧飙升。5月4日当天,全平台发布的信息条数突破了145万条,互动总量达到285,145,381次,这充分表明舆论在意见领袖发声后迅速达到爆发状态。
    \item \textbf{情感倾向分析}:情感分析数据显示,除中性情绪信息外,微博上约有24\%的帖子表现出明显的负面情绪,大部分为愤怒和悲伤。网民对于事件的讨论大多围绕“女友榨取钱财”以及“情感操控”等内容展开,这种强烈的情绪化叙述使得网络争论迅速升级。
    \item \textbf{关键意见领袖的影响}:网红“Yin Shihang 77”的发声成为事件爆发的重要转折点。5月4日凌晨,其发布了支持胖猫家人以及谴责不良情感操控的言论,迅速引发百万级粉丝的共鸣和转发,进一步推动了事件信息的二次扩散。这表明在舆情发酵过程中,关键意见领袖对信息“点燃”作用十分显著,其信息传播效应不可忽视。
    \item \textbf{讨论主题与热点分布}:通过关键词统计与主题分析,可以看出微博讨论主要围绕以下几大主题:
\end{itemize}
\begin{table}[htbp]
\centering
\caption{微博讨论主题分布}
\begin{tabular}{p{3cm}p{6cm}p{4cm}}
\toprule
主题类别 & 讨论内容描述 & 代表性关键词 \\
\midrule
情感操控与网络暴力 & 针对女友涉嫌PUA、情感操控及网暴的激烈讨论 & “PUA”、“榨取”、“网暴” \\
个人悲剧与自杀原因 & 分析自杀背后的经济压力和情感绝望,表达对个人悲剧的同情和质疑 & “自杀”、“绝望”、“压力” \\
媒体报道与信息失实 & 讨论媒体报道的完整性、信息二次传播过程中存在的失真现象,也有网友呼吁理性讨论 & “谣言”、“失真”、“核实” \\
品牌危机公关 & 部分商业品牌因“悼念胖猫”行为被质疑商业化操作,激发讨论品牌责任与危机公关策略 & “道歉”、“危机”、“品牌” \\
\bottomrule
\end{tabular}
\end{table}

\subsection{抖音及其他平台舆情分析}
除了微博之外,抖音、微信、论坛等平台同样作为信息扩散的重要渠道,在此次事件中也展现出一些独特的现象:
\begin{itemize}
    \item \textbf{抖音平台:短视频传播效应}:抖音作为以视频内容为主的平台,其信息传播速度快、覆盖广。虽然具体数据相对微博数据较少,但根据相关报道,抖音上与胖猫事件有关的短视频和讨论帖数量巨大,关键视频内容多次被推荐,形成了新的信息扩散热点。例如,一些网民上传了现场悼念、外卖送餐的视频,这些内容进一步增加了事件的视觉冲击和情感感染力。
    \item \textbf{微信与论坛:深入讨论与观点碰撞}:在微信群聊和论坛中,讨论则相对理性。许多网民从社会结构和性别角色等层面对事件进行探讨,提出对传统父权观念以及舆论群体极化现象的批评。这些讨论虽然未能形成大规模的网络暴力,但依然反映出公众对事件背后深层次社会问题的关注。
    \item \textbf{不同平台讨论氛围的对比}:总体来看,微博平台上议论情绪更为高涨,话题转发和互动量巨大,容易出现极化现象;而抖音、微信、论坛等平台则存在更多长文分析和视频讨论,信息传播节奏较慢,但讨论内容更为深入和理性。各平台之间的信息互动也推动了事件从快速扩散到深度讨论的多层次舆情演变过程。
\end{itemize}

\section{舆情扩散与影响因素分析}
\subsection{关键意见领袖的作用}
网络环境下,关键意见领袖(KOL)在信息扩散中扮演着“催化剂”的角色。以微博平台为例,当“Yin Shihang 77”发布支持胖猫家属的言论后,其粉丝基数迅速放大了信息影响力,使得同一主题在短时间内获得数以百万计的浏览量和转发量。这种现象证明:
\begin{itemize}
    \item 关键意见领袖的言论不仅能够迅速抓住网民情绪,还能在舆情转折期发挥导向作用。
    \item 网红经济与意见领袖效应在悲剧性事件中有双重作用:一方面使事件迅速走红,另一方面也可能将个体悲剧升级为群体情感宣泄和对立冲突。
\end{itemize}

\subsection{媒体报道与信息完整性的影响}
媒体在事件报道中所传递的信息完整性和客观性直接影响公民对事件的认知和情绪表达。
\begin{itemize}
    \item \textbf{官方媒体与民间信息的不一致}:部分媒体报道以官方口径进行信息整合,但大多数情况下是二手数据整合而成,容易导致信息不客观甚至引导负面情绪。数据显示,24\%的官方媒体报道存在明显负面情绪,这表明媒体在传播过程中存在一定程度的信息失真问题。
    \item \textbf{信息过滤与自我审查}:平台自身的内容审核和自我审查机制也对舆情走向产生影响。例如,小红书等平台因内容被政府审查而形成独特的舆论氛围,其过滤机制在一定程度上维护了信息的“正能量”,但同时也引发了网民对信息真实性的质疑。
\end{itemize}

\subsection{危机公关案例的反映}
事件发生后,各商业品牌和企业纷纷介入,部分品牌的危机公关处理方法反映出企业应对舆情时的谨慎态度:
\begin{itemize}
    \item \textbf{品牌危机:茶百道与麦当劳案例}:例如,部分餐饮品牌在悼念胖猫事件中采取的“祭奠”行为本意是表达对逝者的惋惜,但过度商业化行为却引发了网友的强烈反感和“流量反噬”现象,甚至引出网络群体对男女立场的激烈对立。
    \item \textbf{公关回应与信息及时性}:麦当劳越南分部由于不当广告内容被迫公开道歉,显示出在信息敏感时期,企业的沟通方式需要充分考虑公众情感和舆论预期,在反应速度和信息准确性上均不容马虎。
    \item \textbf{品牌公关困境与信息传播}:总体看来,危机公关案例不仅仅反映品牌处置不当的问题,更揭示在网络舆情扩散环境下,企业如何在权衡情感共鸣与商业利益之间找到合理平衡点是一大挑战。
\end{itemize}

\section{数据可视化展示}
\subsection{各平台信息量对比表}
\begin{table}[htbp]
\centering
\caption{各平台信息量对比}
\begin{tabular}{lccc}
\toprule
平台名称 & 信息条数 & 互动量 & 占比 \\
\midrule
微博 & 2,752,528 & 285,145,381 & 43.85\% \\
抖音 & 数据不全 & 大量转发 & - \\
\bottomrule
\end{tabular}
\end{table}

\subsection{微博舆情趋势时序图}
\begin{center}
\begin{tikzpicture}
\begin{axis}[
    xlabel=时间,
    ylabel=信息量(万条),
    xtick=data,
    xticklabels={4-11,4-15,5-2,5-4,5-8,5-19},
    ymin=0,
    legend pos=north west]
\addplot coordinates {
    (0,5) (1,15) (2,80) (3,145) (4,120) (5,30)
};
\legend{微博信息量}
\end{axis}
\end{tikzpicture}
\end{center}

\subsection{情感分布饼图}
\begin{center}
\begin{tikzpicture}
\pie[text=legend]{
    35/正面,
    41/中性,
    24/负面
}
\end{tikzpicture}
\end{center}

\section{讨论与启示}
在对胖猫事件期间各平台舆情变化进行综合分析后,我们可以归纳出以下几方面的讨论与启示:
\begin{itemize}
    \item \textbf{信息传递的速度与舆情极化}:社交媒体平台的去中心化和多对多传播功能使得信息扩散速度快,但同时也容易造成舆论极化。事件发生后关键时刻意见领袖的介入使得原本局部讨论被迅速放大,进而引发大规模情感共鸣与社会对立。未来在遇到类似悲剧时,监管部门需要及时介入,以防止非理性情绪泛滥。
    \item \textbf{情感倾向与信息完整性的关系}:统计数据表明,尽管大部分信息呈现正面或中性情绪,但约24\%的报道存在明显的负面情绪,这反映出媒体报道和个体讨论中存在一定程度的信息缺失和情感渲染。如何确保报道的客观性和信息的完整性,是今后网络信息管理必须解决的重要问题。
    \item \textbf{跨平台信息差异与互动特性}:不同平台由于用户结构和传播机制不同,舆情表现各异。微博作为实时热点信息的主要阵地,其信息互动和转发机制极易形成放大效应;而抖音、微信、论坛等平台则更多以深度讨论和视频展示为主。企业和监管部门在制定舆情应对策略时应充分考虑这些平台的特质,采取有针对性的监测和干预措施。
    \item \textbf{企业危机公关与社会责任}:品牌在面对悲剧时所采取的公关策略不仅影响企业形象,也会对公众情绪产生直接影响。茶百道、麦当劳等品牌因不当举措引发的公关危机显示,企业必须在危机公关过程中把握平衡,避免过度商业化利用悲剧,以免引发公众进一步的反感和对立。
    \item \textbf{数据监控与舆情预警机制的构建}:及时、准确地掌握各平台信息量、互动量及情感分布,对于政府部门和企业及时介入、有效化解舆情危机具有重要意义。构建基于大数据分析的舆情监控平台,通过定量指标和定性内容分析相结合的方式,为后续危机预警提供数据支持,是未来工作的重点方向。
\end{itemize}

\section{结论与主要发现}
经过系统分析胖猫自杀事件期间各平台舆情的变化,我们得出以下主要结论:
\begin{itemize}
    \item \textbf{事件发展呈现明显时间节点特征}:从4月11日事件发生,到5月2日家属揭露,再到5月4日舆情高峰及5月19日警方公布信息,每一阶段均伴随着信息量和互动量的显著变化。这些节点构成了事件扩散的骨架,对理解网络传播规律至关重要。
    \item \textbf{微博平台在舆情扩散中起到核心作用}:微博以其信息传播速度快、用户基数大和互动机制活跃的特点,使得事件在短时间内迅速走红。同时,意见领袖的介入对舆情走向产生了明显的“点燃”效果。
    \item \textbf{情感极化与信息失真问题值得重视}:尽管多数报道倾向于正面或中性,但负面情绪的存在不可忽视。媒体报道的信息完整性和客观性直接影响公众情绪和事件解读,未来需要进一步改善信息传递机制。
    \item \textbf{跨平台特性与企业危机公关互动复杂}:不同平台的信息传播特性不同,使得同一事件在不同场景下呈现出多样化舆情。企业和品牌在危机公关中需要理性回应,避免因情绪化操作而引发更大范围的网络反弹。
\end{itemize}
\begin{table}[htbp]
\centering
\caption{主要发现汇总}
\begin{tabular}{p{5cm}p{8cm}}
\toprule
主要发现 & 说明 \\
\midrule
时间节点明确 & 事件从爆发到平缓存在明显时间切割点 \\
微博核心作用 & 网红发声显著推动舆情极化 \\
情感极化现象普遍 & 负面情绪比例约为24\%,网民情绪集中在愤怒和悲伤 \\
平台特性决定传播效果 & 抖音侧重视频传播,论坛体现深度讨论 \\
企业危机公关需合理 & 品牌需平衡商业利益与公众情感 \\
\bottomrule
\end{tabular}
\end{table}

\section{总结}
本文通过对2024年胖猫自杀事件期间各平台舆情变化的详细分析,构建了一条从事件爆发到舆情平缓的时间线,深入探讨了包括微博、抖音在内的多个社交平台上的信息扩散、互动量和情感倾向。通过数据采集、情感分析和时序对比,揭示了意见领袖在舆情引爆中的关键作用、媒体信息完整性对公众情绪的直接影响以及企业在危机公关中需具备的责任感。

研究结论对网络舆情监管、媒体传播机制以及企业危机公关具有重要启示作用,也为今后应对类似公共事件提供了数据支持和理论参考。主要启示包括:
\begin{itemize}
    \item 加强对关键节点的监控与事件快速干预,防止情绪极化和误导性信息扩散;
    \item 提高媒体报道的客观性及信息核实力度,避免因信息失真而引发不必要的群体对立;
    \item 企业在危机公关中应及时回应并采取理性、公正的态度,防止商业化操作引发的公众反感;
    \item 进一步构建跨平台、全时段的舆情监控系统,利用大数据和人工智能技术实时掌握网络情绪变化,确保公共信息传播安全和社会稳定。
\end{itemize}
总之,胖猫事件不仅仅是一起个体悲剧,它揭示了当前网络时代下信息爆炸和情感极端化的普遍问题,值得社会各界高度重视。未来效仿此类研究能为政府、企业和社会公众提供一种新型的舆情引导和危机处理模式,从而在信息化社会中更好地维护公共安全和社会和谐。

\section*{参考文献}
\begin{enumerate}
    \item 腾讯新闻. (2025-03-16). 《心疼那90多吨粮食!》\url{https://new.qq.com/rain/a/20250316A03ZCO00}.
    \item 搜狐网. (2025-03-17). 《AI深度解析胖猫事件》\url{https://www.sohu.com/a/872071925_122066679}.
    \item 腾讯新闻. (2025-03-16). 《哀渝州——胖猫事件复盘》\url{https://new.qq.com/rain/a/20250316A00ID900}.
    \item 搜狐网. (2025-03-16). 《胖猫事件央视真相还原》\url{https://www.sohu.com/a/871575523_121654491}.
    \item 网易新闻. (2025-03-15). 《央视还原胖猫事件真相》\url{https://www.163.com/dy/article/JQNMB62A0556AP6K.html}.
    \item 搜狐网. (2025-03-17). 《史上最长警情通报都拦不住谣言》\url{https://www.sohu.com/a/871957783_121106842}.
    \item 搜狐网. (2024-05-09). 《解析重庆胖猫事件中的舆论》\url{https://www.sohu.com/a/777640801_121124034}.
    \item 央视网.(2025-05-19)《“胖猫”事件调查情况》
    \url{https://news.cctv.com/2024/05/19/ARTIyvF3Zjk2KqrAvcx4LpV3240519.shtml}.
\end{enumerate}

\end{document}